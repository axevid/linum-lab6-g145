\documentclass[10pt, a4paper]{article}
\usepackage{amsmath}
\usepackage{hyperref}
%\usepackage{subfig}
\hypersetup{colorlinks=true, urlcolor= blue}
\usepackage{graphicx}
\usepackage[swedish]{babel}
\usepackage[T1]{fontenc}
\usepackage[utf8]{inputenc}
\usepackage[margin = 2.0cm]{geometry}
%\usepackage{dpfloat}
%\usepackage{natbib}
%\usepackage{multicol}
%\usepackage{multirow}
%\usepackage{setspace}
%\usepackage{multicol}
%{nomencl} 
\usepackage{subfig}
\usepackage{color}
%\usepackage{float}
%\usepackage{balance}
%\usepackage{paralist}
\usepackage{listings}
\usepackage{framed}
%\bibpunct{[}{]}{;}{n}{,}{,}
%\renewcommand{\d}{\mathrm{d}}
%\newcommand{\T}{\mathrm{T}}
%\makenomenclature
%\renewcommand*{\url}[1]{\href{#1}{#1}}
%\newcommand{\HRule}{\rule{\linewidth}{0.5mm}}
%\newcommand{\hRule}{\rule{\linewidth}{0.2mm}}
\newcommand{\eqr}[1]{Eq. \eqref{eq:#1}}

\author{Axel Vidmark 8710171417 \href{mailto:axel.vidmark@gmail.com}{\texttt{axel.vidmark@gmail.com}}}
\title{Linux som utvecklingsmiljö \\ Övning 6 - Bibliotek}
\begin{document}
\maketitle

% \section*{Collaboration}
% All tasked were solved by me alone but a comparison of results were made with Fredrik Backman.

\section*{Del 1}\label{sec:del1}

Uppgift:
\begin{enumerate}
Redovisa en beskrivning av det egna biblioteket och applikationen vad gäller:

Funktion och användning.
En algoritmbeskrivning.
Hur du kompilerat och testat det i ett eget program. Beskriv vilka kommandon du använder för att kompilera biblioteket och hur du sedan använder biblioteket i ditt huvudprogram.
Beskriv de växlar du använt till kommandona ovan och vad var och en av dessa har för funktion.
\end{enumerate}

\vspace{10pt}
Svar:

\begin{verbatim}
axvi@axvi-VirtualBox:/tmp$ sudo apt-get install git


\end{verbatim}


\cleardoublepage

\section*{Del 2}\label{sec:del2}

Uppgift:
Skriv en Makefile med dessa regler:

lib, för att bygga enbart biblioteket.
all, För att bygga både programmet och biblioteken där biblioteken läggs i en egen katalog, lib,  under den man är jus nu, tex /home/bl/electro/lib/. Här ska programmet länkas för att använda de lokala biblioteken. OBS! Ni får inte temporärt ändra libsökvägarna i LD_LIBRARY_PATH!
install. Här kopierar du både programmet och biblioteken till lämpliga kataloger (tex /usr/bin/ och /usr/lib/) och länkar så att programmet använder de publika biblioteken.
Redovisa en beskrivning av hur du länkat in alla 3 bibliotek och använt det i ett huvudprogram enligt ovan. Beskriv också vilka du jobbat med.

Beskriv vilka kommandon och växlar du använder för att kompilera biblioteken och hur du sedan använder biblioteken i ditt huvudprogram.
 
\begin{verbatim}
> mkdir bjorne_lindberg
> echo "Hello svn world" > bjorne_lindberg/bl.txt
> svn add bjorne_lindberg
> svn commit -m "Lagt till en katalog med fil"
\end{verbatim}

\cleardoublepage

\section*{Del 3}\label{sec:del3}

Uppgift:
Skriv en kort beskrivning av hur samarbetet gått samt en reflektion över vad som krävs för att ett sådant här arbete ska fungera även i större skala med betydligt större skara utvecklare och större kodbas.
 





% \appendix
% \section{MATLAB\label{app:matlab}}
% \lstset{language=Matlab
% rulesepcolor=\color{blue},basicstyle=\footnotesize}
% \begin{framed}
% \lstinputlisting{ha4axel.m} 
% \end{framed}

\end{document}